\section*{\centering \Huge \textbf{Chapter 6: TESTING \& DEPLOYMENT}}
\vspace{2cm}
\addcontentsline{toc}{chapter}{Chapter 6: Testing, Documentation, Deployment and Hosting}

\section{Testing Strategy}

\subsection{Unit Testing}

\textbf{Backend Testing with Pytest:}

\begin{lstlisting}[language=Python]
# tests/test_bayesian_wilson.py
import pytest
from services.small_sample_analytics import SmallSampleAnalytics

def test_bayesian_dampening_prevents_100_percent():
    """Verify 1 success in 1 attempt != 100% confidence"""
    ssa = SmallSampleAnalytics()
    result = ssa.calculate_aes(successes=1, total=1)
    
    assert result['point_estimate'] < 1.0  # Dampened by prior
    assert result['conservative_score'] < 0.5  # Wilson lower bound
    assert result['method'] == 'bayesian_wilson_hybrid'

def test_wilson_handles_zero_successes():
    """Ensure 0/1 doesn't crash or return NaN"""
    ssa = SmallSampleAnalytics()
    result = ssa.calculate_aes(successes=0, total=1)
    
    assert result['point_estimate'] >= 0.0
    assert result['confidence_interval'][0] >= 0.0
    assert result['confidence_interval'][1] <= 1.0

def test_large_sample_converges_to_frequentist():
    """With n>50, Bayesian should approximate frequentist"""
    ssa = SmallSampleAnalytics()
    result = ssa.calculate_aes(successes=70, total=100)
    
    frequentist_estimate = 70 / 100
    assert abs(result['point_estimate'] - frequentist_estimate) < 0.05
\end{lstlisting}

\subsection{Integration Testing}

\begin{lstlisting}[language=Python]
# tests/test_decision_engine.py
import pytest
from app import create_app
from models.base import db

@pytest.fixture
def client():
    app = create_app('testing')
    with app.test_client() as client:
        with app.app_context():
            db.create_all()
            yield client
            db.drop_all()

def test_get_advice_endpoint(client):
    """Test complete decision workflow"""
    # 1. Register user
    response = client.post('/api/v1/auth/register', json={
        'username': 'test_farmer',
        'password': 'secure123',
        'phone_number': '+21612345678',
        'governorate': 'Sfax'
    })
    assert response.status_code == 201
    
    # 2. Login
    response = client.post('/api/v1/auth/login', json={
        'username': 'test_farmer',
        'password': 'secure123'
    })
    assert response.status_code == 200
    token = response.json['token']
    
    # 3. Request advice
    response = client.post('/api/v1/decisions/get-advice',
        headers={'Authorization': f'Bearer {token}'},
        json={
            'crop_id': 1,
            'governorate': 'Sfax',
            'seedling_cost': 0.5,
            'market_price': 2.5
        }
    )
    assert response.status_code == 200
    assert 'decision' in response.json
    assert response.json['decision']['action'] in [
        'PLANT_NOW', 'WAIT', 'NOT_RECOMMENDED'
    ]
\end{lstlisting}

\subsection{API Testing with Insomnia}

All major endpoints were validated using Insomnia REST client:

\begin{figure}[H]
\centering
\includegraphics[width=0.8\textwidth]{figures/insomnia_testing.png}
\caption{Insomnia REST Client Testing Organization}
\label{fig:insomnia_testing}
\end{figure}

\textbf{Test Collections:}
\begin{itemize}
    \item \textbf{Authentication}: Register, login, token refresh
    \item \textbf{Public Data}: Crops, governorates, periods
    \item \textbf{Decision Engine}: Get advice, record outcome
    \item \textbf{Analytics}: Personal dashboard, regional benchmarks
    \item \textbf{Admin}: Moderation, content management
\end{itemize}

\section{Performance Testing}

\subsection{Load Testing Results}

\begin{table}[H]
\centering
\small
\begin{tabularx}{\textwidth}{@{}lXXX@{}}
\toprule
\textbf{Endpoint} & \textbf{Concurrent Users} & \textbf{Avg Response Time} & \textbf{Error Rate} \\
\midrule
GET /api/v1/crops & 1000 & 45ms & 0\% \\
POST /api/v1/decisions/get-advice & 500 & 280ms & 0.2\% \\
GET /api/v1/analytics/personal & 200 & 150ms & 0\% \\
POST /api/v1/auth/login & 100 & 120ms & 0\% \\
\bottomrule
\end{tabularx}
\caption{Load Testing Results (Locust Framework)}
\label{tab:load_testing}
\end{table}

\section{API Documentation}

\subsection{OpenAPI 3.0 Specification}

The project features comprehensive API documentation via \texttt{flask-smorest}:

\begin{figure}[H]
\centering
\includegraphics[width=0.85\textwidth]{figures/swagger_ui.png}
\caption{Swagger UI: Interactive API Documentation}
\label{fig:swagger_ui}
\end{figure}

\textbf{Features:}
\begin{itemize}
    \item Live interactive UI at \texttt{/swagger-ui/}
    \item Strictly defined schemas for all endpoints
    \item Testable requests directly from browser
    \item Auto-generated client SDKs (Python, JavaScript)
    \item Request/response examples for every endpoint
\end{itemize}

\section{Deployment}

\subsection{Containerization with Docker}

\textbf{Benefits of Docker Deployment:}
\begin{itemize}
    \item \textbf{Environment Determinism}: Identical behavior in dev/staging/production
    \item \textbf{Dependency Isolation}: No conflicts between Python/Node versions
    \item \textbf{Scalability}: Easy horizontal scaling with Docker Swarm/Kubernetes
    \item \textbf{Rollback Safety}: Instant rollback to previous container version
\end{itemize}

\subsection{Deployment Workflow}

\begin{enumerate}
    \item \textbf{Build Images}:
    \begin{lstlisting}[language=bash]
docker-compose build
    \end{lstlisting}
    
    \item \textbf{Run Containers}:
    \begin{lstlisting}[language=bash]
docker-compose up -d
    \end{lstlisting}
    
    \item \textbf{Verify Health}:
    \begin{lstlisting}[language=bash]
curl http://localhost:5000/api/v1/health
    \end{lstlisting}
    
    \item \textbf{View Logs}:
    \begin{lstlisting}[language=bash]
docker-compose logs -f backend
    \end{lstlisting}
\end{enumerate}

\section{Hosting}

\subsection{Production Hosting on Render}

The AgriDecision-TN platform is deployed on \textbf{Render}, which provides:

\begin{itemize}
    \item \textbf{Automated CI/CD}: Deploys automatically from GitHub commits
    \item \textbf{Managed PostgreSQL}: Fully managed database with automated backups
    \item \textbf{Free SSL/TLS}: Automatic HTTPS certificates
    \item \textbf{Global CDN}: Fast content delivery for frontend assets
    \item \textbf{Zero Downtime Deploys}: Blue-green deployment strategy
\end{itemize}

\textbf{Live URL:} \url{https://agridecision-tn.onrender.com}

\subsection{Monitoring \& Logging}

\begin{itemize}
    \item \textbf{Application Logs}: Centralized logging with Python \texttt{logging} module
    \item \textbf{Error Tracking}: Sentry integration for real-time error alerts
    \item \textbf{Performance Metrics}: Prometheus + Grafana dashboards
    \item \textbf{Uptime Monitoring}: UptimeRobot checks every 5 minutes
\end{itemize}

\section{Continuous Integration}

\subsection{GitHub Actions Workflow}

\begin{lstlisting}[language=yaml]
# .github/workflows/ci.yml
name: CI/CD Pipeline

on:
  push:
    branches: [ main ]
  pull_request:
    branches: [ main ]

jobs:
  test:
    runs-on: ubuntu-latest
    
    steps:
    - uses: actions/checkout@v3
    
    - name: Set up Python
      uses: actions/setup-python@v4
      with:
        python-version: '3.9'
    
    - name: Install dependencies
      run: |
        cd backend
        pip install -r requirements.txt
    
    - name: Run tests
      run: |
        cd backend
        pytest --cov=. --cov-report=xml
    
    - name: Upload coverage
      uses: codecov/codecov-action@v3
      with:
        file: ./backend/coverage.xml

  deploy:
    needs: test
    runs-on: ubuntu-latest
    if: github.ref == 'refs/heads/main'
    
    steps:
    - name: Deploy to Render
      run: |
        curl -X POST ${{ secrets.RENDER_DEPLOY_HOOK }}
\end{lstlisting}
